\documentclass[12pt]{article}
\usepackage[utf8]{inputenc}
\usepackage[T1]{fontenc}
\usepackage[brazilian]{babel}
\usepackage{verbatim}

\title{Lista de Exercícios - Memória}
\author{Rafael Gonçalves de Oliveira Viana}
\date{4º semestre de 2017}

\begin{document}

\maketitle

\begin{enumerate}
\item
Em um sistema sem abstração de memória com múltiplos processos em
memória, porque é necessário uma forma de realocação ?
\subparagraph{Resposta}
Por que cada programa que esta rodando foi compilado como constantes, e para que um novo processo entre em execução é precisso liberar o endereço de memoria, salvando o processo anterior em disco(relaocando ele em disco), e então permitindo que o novo processo manipule a memória, sem prejudicar o programa anterior porém isso e custoso para o S.O.
\item
Qual forma de realocação (estática ou dinâmica) é mais fácil de ser
usada pelo S.O.? Cite um motivo pelo qual o S.O. não consiga utilizar
o método mais fácil.
\subparagraph{Resposta}
Realoção dinâmica o S.O  e quem é que determina os valores do registrador-base e do registrador-limite, que faz o calculo de posicionamento diferente na memória física. Um dos principais motivos do sistema operacional não a utilizar é porque nesse esquema como o processo está totalmente "carregado"em memória, se o número de processos for muito grande, manter todos eles em memória estouraria sua capacidade, precisando então fazer swap a cada vez que um novo processo não "coubesse" em memória. Com a memória virtual, você só precisa manter partes do processo em memória, e não é necessário estar em partes consecutivas. 

\item
Suponha que o S.O. utilize realocação estática dos processos e a
memória está muito fragmentada. Em um dado momento, o S.O. decide
compactar a memória, movendo os segmentos de processos para o início
da memória. Esta operação seria mais eficiente se fosse usada
realocação dinâmica? Porque?
\subparagraph{Resposta}
	Sim. Deixaria menos espaços fragmentados, sendo assim os espaços inutilizados seram menore. Quando as trocas de processos deixam "muitos" espaços vazios na memória, é possivel combiná-las todos em um único espaço contíguo de memória, movendo-os, o máximo possível, para os endereços mais baixos. Essa técnica é denominada compactção de memória. Ela geramente não é usada em virtude do tempo de processamento necessário.

\item
Suponha que três processos executam e necessitam, respectivamente, de
1GiB, 0,5GiB e 1,5GiB de memória. A máquina em que executam tem apenas
2GiB de memória. Qual forma de gerenciamento de memória seria mais
eficiente nesta situação, sendo impossível alterar o código dos
processos?
\subparagraph{Resposta}
Os processos 2 e 3 são executados juntos, quando termina o processo 1 e colocado em memoria.

\item
Suponha que o S.O. implemente paginação de memória e apenas 10\% das
páginas de cada processo estejam na memória e todas demais em
disco. Se houverem 10 processos prontos que ainda tem muito o que
executar, é possível que executem sem que as faltas de páginas se
tornem um gargalo de desempenho? Explique.
\subparagraph{Resposta}

Sim, Se um programa estiver esperando por outra parte de si próprio ser carregada na memória, a CPU poderá ser dada a outro processo sem que ele fique bloqueado dando continuidade aos demais processos que tambem serão carregados do disco.

\item
Suponha que o acesso ao endereço de memória $1056 = 10000100000_2$
deva ser traduzido pela MMU em um endereço real de memória. Suponha
ainda que a página $1$ está na moldura $5$ (a página $0$ é a
primeira). Calcule o endereço real para páginas de $997$ bytes e para
páginas de $1024$ bytes. Você pode calcular em binário se achar mais
fácil.
\subparagraph{Resposta}





\item
Suponha que você está projetando um sistema de paginação para uma
máquina de 32 bits cujo endereço das páginas tem 14 bits e cada
entrada da tabela de páginas deve conter 6 bits de status além do
número da moldura. Se você usar uma única tabela de páginas, qual é o
tamanho mínimo da tabela?
\subparagraph{Resposta}
Sobram 17 bits, o tamanha de página seria de 2exp17

\item
Agora suponha que você está usando tabelas de página em dois níveis e
que o endereço da página seja subdividido em duas partes desiguais. A
primeira parte, usada para endereçar a tabela de primeiro nível tem 6
bits. A segunda parte tem 12 bits e endereça as tabelas de segundo
nível. Qual o tamanho máximo da tabela de páginas (sobram 14 bits para
o deslocamento) considerando que cada entrada precisa de 6 bits de
status?
\subparagraph{Resposta}
32  bits divididos em um campo PT1 de 6 bits, um campo PT2 de 10 bits e um campo Deslocamento de 14 bits, as páginas são de tamanho 16 KB. Os outros dois campos tem conjuntamente 16bits o que possibilita um total de 2exp20
\item
A ideia por trás do endereçamento em dois níveis é usar menos memória
para armazenar a tabela de páginas. Como isso é possível no caso acima
descrito?
\subparagraph{Resposta}
 Simples a ideia é evitar manter na memória todas tabelas de página o tempo todo, fazendo ela carregar apenas o tabela principal e se necessario trazer para memória a tabela secundaria.

\item
Suponha que a TLB tenha 32 entradas. Qual o tamanho máximo do espaço
de endereçamento que a TLB pode traduzir se o tamanho da página é de
4KiB?

\subparagraph{Resposta}
\item
Suponha uma máquina de 64 bits com páginas de 4KiB e um total de 8GiB
de memória física. O gerenciador de memória do S.O. utiliza uma tabela
de páginas invertida que armazena, em cada linha, o número da página,
o número do processo (14 bits) e mais 6 bits de estado. Qual é o tamanho
da tabela?
\subparagraph{Resposta}
\item
Qual é o maior problema com as tabelas de páginas invertidas? Indique
algumas maneiras de minimizá-lo.
\subparagraph{Resposta}
\item
Suponha que uma falta de página leve 10ms para ser resolvida (trazer a
página para a memória) e que uma instrução normalmente executa em 1ns
(se não houver faltas de páginas). Um determinado programa leva 3
minutos para executar e sofre um total de 15.000 faltas de página. Seu
dono decide, então, dobrar a quantidade de memória da máquina e
observa que o número de faltas de página reduziu-se pela
metade. Quanto tempo o programa demora para executar na nova
configuração?
\subparagraph{Resposta}
\item
Suponha uma máquina com apenas 4 molduras de página e um processo com
8 páginas. Inicialmente todas as molduras estão livres e o processo
começa a executar, fazendo acesso às seguintes páginas:
\[ 1, 1, 0, 2, 3, 2, 5, 0, 4, 0, 2, 6, 2, 0, 3, 5, 3, 2, 7, 3 \]
Simule a execução do algoritmo FIFO (mais antigo sai) e do
relógio. Isto é, determine qual página é removida da memória para dar
lugar a cada página ausente.
\subparagraph{Resposta}
\item
Retome a situação anterior e suponha que a cada 5 referências à
memória ocorre um ciclo de atualização de contadores. Simule os
algoritmos NFU e WSClock.
\subparagraph{Resposta}
%% \item
%% Suponha que uma máquina possui 12KiB de memória e está utilizando
%% segmentação pura. Quando um segmento deve ser carregado do disco em
%% memória, um algoritmo é executado para encontrar uma lacuna grande o
%% suficiente e o segmento é copiado para o começo da lacuna. Suponha que
%% um processo tenha 8 segmentos, cujos tamanhos são 2, 3, 4, 1, 2, 2, 3,
%% 2, respectivamente. Quando não existe lacuna grande o suficiente, o
%% sistema remove o menor segmento tal que a lacuna formada é
%% suficientemente grande. Se nenhum segmento satisfizer a condição, o
%% segmento mais antigo é removido e o algoritmo é reiniciado.

%% Simule a execução do algoritmo da melhor escolha para a seguinte
%% sequência de acessos a segmentos: 0, 2, 1, 4, 6, 7, 3
\end{enumerate}
\end{document}
